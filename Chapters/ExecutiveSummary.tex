The Tokai to Kamioka (T2K) experiment is currently making critical
measurements of $\deltacp$ and other properties that describe neutrino
oscillations. Using a long-baseline of 295 km, a beam of muon-type
neutrinos (or antineutrinos) are detected to change flavor at the
Super-Kamiokande (SK) detector. By counting the number of oscillated
neutrinos and antineutrinos, measurements on the oscillation parameters
can be made.

The largest sources of systematic uncertainty in the oscillation analysis
are related to the flux of neutrinos and the neutrino-nucleus cross
section. Using data collected at the T2K near detector, ND280, stronger
constraints are placed on those systematic uncertainties in the analysis.
The process for constraining these uncertainties uses a binned maximum
likelihood fit based on neutrino interaction topology. This technote
provides a independent measurement of the constraint using data collected
using the T2K pi-zero detector ($\pod$) in ND280.

The technote is organized as follows. The binned likelihood procedure
is described in \prettyref{chap:BANFF-Likelihood} followed by the
data selections described in \prettyref{chap:P0DSelections}. The
systematic uncertainties in the analysis and the fit parameterization
are described in \prettyref{chap:P0DinBANFF}. The validation of the
fit procedure will be shown in \prettyref{chap:Fitter-Validation},
which is followed by the results of the likelihood fit shown in \prettyref{chap:Fitter-Results}.
Finally, \prettyref{chap:Discussion} provides some concluding remarks
and discussion on the results and possible analysis improvements.
